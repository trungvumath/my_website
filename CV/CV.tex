% --- LaTeX CV Template - S. Venkatraman ---

% --- Set document class and font size ---

\documentclass[letterpaper, 11pt]{article}

% --- Package imports ---

\usepackage{hyperref, enumitem, longtable, amsmath, array}

% --- Page layout settings ---

% Set page margins
\usepackage[left=0.7in, right=0.8in, bottom=.8in, top=0.8in, headsep=0in, footskip=.2in]{geometry}

% Set line spacing
\renewcommand{\baselinestretch}{1.2}

% --- Page formatting settings ---

% Set link colors
\usepackage[dvipsnames]{xcolor}
\hypersetup{colorlinks=true, linkcolor=MidnightBlue, urlcolor=MidnightBlue}

% Set font to Libertine, including math support
\usepackage{libertine}
\usepackage[libertine]{newtxmath}

% Remove page numbering
\pagenumbering{gobble}

% Define font size and color for section headings
\newcommand{\headingfont}{\Large\color{Red}}

% --- CV section settings ---

% Note: each section of this table (Education, Awards, Publications etc.) is 
% stored in a two-column table. The left-hand column is narrow (1 inch) and is 
% meant to store dates. The right-hand column is wide (5.2 inches) and stores 
% the main text.  Sections in which each entry might have multiple lines 
% (e.g., Education) are stored in a 'SectionTable' environment). Sections in 
% which each entry might just have one line are stored in a 'SectionTableSingleSpace'
% environment. The only difference between the two environments is the line 
% spacing between each entry. Both environments take one argument, which is the
% title of the section. See main document for how these environments are used.

% Define settings for left-hand column in which dates are printed
\newcolumntype{R}{>{\raggedleft}p{1in}}

% Define 'SectionTable' environment
\newenvironment{SectionTable}[1]{
	\renewcommand*{\arraystretch}{1.7}
	\setlength{\tabcolsep}{10pt}
	\begin{longtable}{Rp{5.2in}} & #1 \\}
	{\end{longtable}\vspace{-.3cm}}

% Define 'SectionTableSingleSpace' environment
\newenvironment{SectionTableSingleSpace}[1]{
	\renewcommand*{\arraystretch}{1.2}
	\setlength{\tabcolsep}{10pt}
	\begin{longtable}{Rp{5.2in}} & #1 \\[0.6em]}
	{\end{longtable}\vspace{-.3cm}}

% --- Document starts here ---

\begin{document}
	
	% --- Name and contact information ---
	
	\begin{SectionTable}{\Huge Trung Vu} & 
		hvu@illinois.edu   $\;\boldsymbol{\cdot}\;$ 
		www.yourwebsite.com $\;\boldsymbol{\cdot}\;$ 
		507-581-2213 \newline
		Citizenship: Vietnam
	\end{SectionTable}
	
	% --- Section: Research interests ---
	
	\begin{SectionTable}{\headingfont Research interests}
		& Algebraic combinatorics, cluster algebra, combinatorial aspects of vertex models, exactly solved models and integrable systems
	\end{SectionTable}
	
	% --- Section: Education ---
	
	\begin{SectionTable}{\headingfont Education}
		2020 -- Present & 
		\textbf{University of Illinois at Urbana-Champaign} -- Urbana, Illinois \newline
		PhD in Mathematics \newline 
		Mentors: Professors A, B. %\textit{GPA: 3.89}. 
		\\
		
		2016 -- 2020 & 
		\textbf{St. Olaf College} -- Northfield, Minnesota \newline
		BA in Mathematics with concentration (minor) in Neuroscience \newline 
		%Mentors: Professors C, D. \textit{GPA: X.YZ}. 
		\\

		
		% --- Un-comment the next few lines if you want to include some courses you've taken ---
		
		%& \textbf{Selected coursework}
		%\begin{itemize}[itemsep=0pt, leftmargin=*]
		%\item \textit{Statistics}: Asymptotic statistics, Mathematical statistics, Functional data analysis, High-dimensional statistics, Information theory
		%\item \textit{Mathematics}: Measure theory, Functional analysis, Measure-theoretic probability with martingales
		%\end{itemize}
		
	\end{SectionTable}
	
	% --- Section: Awards, scholarships, etc ---
	
	\begin{SectionTableSingleSpace}{\headingfont Honors and scholarships}
		2018 & 
		Pi Mu Epsilon Mathematical Honor Society \\
		
		2019 &
		\href{https://wp.stolaf.edu/curi/summer-undergraduate-research}{Steen Fellowship} (St. Olaf College) \newline 
		\textit{\$4,170  to fund independent summer research project}
		\\
		

	\end{SectionTableSingleSpace}
	
	% --- Section: Publications ---
	
	\begin{SectionTable}{\headingfont Publications} 
		2017 & 
		\textbf{Matrix square roots of polynomials} \newline
		Kosmas Diveris, Trung Vu \newline
		\textit{Pi Mu Epsilon Journal}. \\
		
		%2020 & 
		%\textbf{Title of your second most recent research paper} \newline
		%First author, second author, third author, fourth author. \newline
		%\textit{Journal of something or the other}. \\
		
		%2020 & 
		%\textbf{Title of your third most recent research paper} \newline
		%First author, second author, third author, fourth author. \newline
		%\textit{Journal of something or the other}. \\
		
		%2019 & 
		%\textbf{Title of your fourth most recent research paper} \newline
		%First author, second author, third author, fourth author. \newline
		%\textit{Journal of something or the other}.
	\end{SectionTable}
	
	% --- Section: Research experience ---
	
	\begin{SectionTable}{\headingfont Research experience}
		& \textbf{At. St. Olaf College} \\
		Fall 2017 -- Spring 2019  &
		\textbf{Pupillometry and Auditory Cognition in Normal Hearing Listeners, Hearing Impaired Individuals and Cochlear Implant Users} \newline
		Mentors: Professor Jeremy Loebach (St. Olaf College). \newline
		Project investigating auditory and neurocognitive mechanisms that givse rise to accurate speech perception in a variety of listening environments in normal hearing, hearing impaired and cochlear implant users. Responsibility include testing participants in a multipart auditory neurocognitive battery, setting up and running the eye tracker for pupillometry measurements, helping condition and analyzing data. \\
		
		Fall 2017-Sping 2020 &
		\textbf{Free field sound localization using the SoLoArc Project} \newline
		Mentors:  Professor Jeremy Loebach (St. Olaf College). \newline
		Project focuses on using SoLoArc (Sound Localization Arc) – a student-made portable sound localization apparatus – to test the ability to localize sound in horizontal space using interaural time difference (ITD) and interaural level difference (ILD), and in vertical space using head-related transfer functions and filtering. We developed the graphical user interface (GUI) for the SoLoArc. The GUI gave students who have had limited coding experience in MATLAB easily access and use the device, allowing it to be used in classes with less supervision. We designed and improved the automated pointing system using a potentiometer and a servo for more precise sound indication.
		\\
	\end{SectionTable}
	
	% --- Section: Teaching experience ---
	
	\begin{SectionTable}{\headingfont Teaching experience}
		& \textbf{At. St. Olaf College} \\
		Fall 2017 & 
		\textbf{Teaching assistant, Chem 121: Course name here (University)} \newline
		Topics and description of your responsibilities. Aliquam volutpat est vel massa. Sed dolor lacus, imperdiet non, ornare non, commodo eu, neque. \newline
		\textit{Average student rating: X/5.} \\
		
		Spring 2020 & 
		\textbf{Teaching assistant, MATH 234: Course name here (University)} \newline
		Topics and description of your responsibilities. Aliquam volutpat est vel massa. Sed dolor lacus, imperdiet non, ornare non, commodo eu, neque. \newline
		\textit{Average student rating: X/5.}
	\end{SectionTable}
	
	% --- Section: Industry experience ---
	
%	\begin{SectionTable}{\headingfont Industry experience}
%		Summer 2020 &
%		\textbf{Name of company (Title of job or internship)} -- City, State \newline
%		Description of your responsibilities. Integer pretium semper justo. Proin risus. Nullam id quam. Nam neque. Phasellus at purus et lib ero lacinia dictum.  \\
		
%		Summer 2019 &
%		\textbf{Name of company (Title of job or internship)} -- City, State \newline
%		Description of your responsibilities. Integer pretium semper justo. Proin risus. Nullam id quam. Nam neque. Phasellus at purus et lib ero lacinia dictum.  \\
		
%		Summer 2018 &
%		\textbf{Name of company (Title of job or internship)} -- City, State \newline
%		Description of your responsibilities. Integer pretium semper justo. Proin risus. Nullam id quam. Nam neque. Phasellus at purus et lib ero lacinia dictum.  \\
%	\end{SectionTable}
	
	% --- Section: Talks and tutorials ---
	
	\begin{SectionTable}{\headingfont Talks and tutorials}
		Month Year &
		Title of your most recent presentation \newline
		\textit{Name of conference, workshop, seminar, etc., or a description} \\
		
		Month Year &
		Title of your second most recent presentation \newline
		\textit{Name of conference, workshop, seminar, etc., or a description} \\
		
		Month Year &
		Title of your third most recent presentation \newline
		\textit{Name of conference, workshop, seminar, etc., or a description} \\
	\end{SectionTable}
	
	% --- Section: Mentorship and service ---
	
	\begin{SectionTable}{\headingfont Mentorship and service}
		Month Year -- Present &
		\textbf{Title of organization you are in (Name of your role)} \newline
		Description of your responsibilities. Integer pretium semper justo. Proin risus. Nullam id quam. Nam neque. Phasellus at purus et lib ero lacinia dictum. \\
		
		Month Year -- Month Year &
		\textbf{Title of organization you were in (Name of your role)} \newline
		Description of your responsibilities. Integer pretium semper justo. Proin risus. Nullam id quam. Nam neque. Phasellus at purus et lib ero lacinia dictum. \\
	\end{SectionTable}
	
	% --- Section: Professional society memberships ---
	
	\begin{SectionTable}{\headingfont Professional memberships}
		Year -- Present &
		Name of professional society \newline
		\textit{Short description or conferences you attended.} \\
		
		Year -- Present &
		Name of professional society \newline
		\textit{Short description or conferences you attended.} \\
	\end{SectionTable}
	
	\begin{SectionTable}{\headingfont Technical skills}
		& \textbf{Programming languages} \newline
		Proficient in: Python, MATLAB, R, HTML, CSS, \LaTeX \newline
		Familiar with: C++, Julia\\
		
		& \textbf{Software} \newline
		Proficient in: Sage, Mathematica, Macaulay2 \newline
		Familiar with: Tensor Flow, Arduino packages from MATLAB, Pupil Labs (Eye-tracking devices software) \\
		
		& \textbf{Languages} \newline
		English (fluent), Vietnamese (fluent)
	\end{SectionTable}
	
	% --- Section: Other interests/hobbies ---
	
%	\begin{SectionTable}{\headingfont Other interests}
%		& 
%	\end{SectionTable}
	
	% --- End of CV! ---
	
\end{document}