% --- LaTeX CV Template - S. Venkatraman ---

% --- Set document class and font size ---

\documentclass[letterpaper, 11pt]{article}

% --- Package imports ---

\usepackage{hyperref, enumitem, longtable, amsmath, array}

% --- Page layout settings ---

% Set page margins
\usepackage[left=0.7in, right=0.8in, bottom=.8in, top=0.8in, headsep=0in, footskip=.2in]{geometry}

% Set line spacing
\renewcommand{\baselinestretch}{0.8}

% --- Page formatting settings ---

% Set link colors
\usepackage[dvipsnames]{xcolor}
\hypersetup{colorlinks=true, linkcolor=MidnightBlue, urlcolor=MidnightBlue}

% Set font to Libertine, including math support
\usepackage{libertine}
\usepackage[libertine]{newtxmath}

% Remove page numbering
\pagenumbering{gobble}

% Define font size and color for section headings
\newcommand{\headingfont}{\Large\color{Red}}

% --- CV section settings ---

% Note: each section of this table (Education, Awards, Publications etc.) is 
% stored in a two-column table. The left-hand column is narrow (1 inch) and is 
% meant to store dates. The right-hand column is wide (5.2 inches) and stores 
% the main text.  Sections in which each entry might have multiple lines 
% (e.g., Education) are stored in a 'SectionTable' environment). Sections in 
% which each entry might just have one line are stored in a 'SectionTableSingleSpace'
% environment. The only difference between the two environments is the line 
% spacing between each entry. Both environments take one argument, which is the
% title of the section. See main document for how these environments are used.

% Define settings for left-hand column in which dates are printed
\newcolumntype{R}{>{\raggedleft}p{1in}}

% Define 'SectionTable' environment
\newenvironment{SectionTable}[1]{
	\renewcommand*{\arraystretch}{1.7}
	\setlength{\tabcolsep}{10pt}
	\begin{longtable}{Rp{5.2in}} & #1 \\}
	{\end{longtable}\vspace{-.3cm}}

% Define 'SectionTableSingleSpace' environment
\newenvironment{SectionTableSingleSpace}[1]{
	\renewcommand*{\arraystretch}{1.2}
	\setlength{\tabcolsep}{10pt}
	\begin{longtable}{Rp{5.2in}} & #1 \\[0.6em]}
	{\end{longtable}\vspace{-.3cm}}

% --- Document starts here ---

\begin{document}
	
	% --- Name and contact information ---
	
	\begin{SectionTable}{\Huge Trung Vu} & 
		hvu@illinois.edu   $\;\boldsymbol{\cdot}\;$ 
		htrungvu.com 
		$\;\boldsymbol{\cdot}\;$ 
		507-581-2213 \newline
		Citizenship: Vietnam
	\end{SectionTable}
	
	% --- Section: Research interests ---
	
	\begin{SectionTable}{\headingfont Research interests}
		& Algebraic combinatorics, cluster algebra, exactly solved models, integrable lattice systems, representation theory 
	\end{SectionTable}
	
	% --- Section: Technical Skills ---
		\begin{SectionTable}{\headingfont Technical skills}
		& \textbf{Programming and markdown languages} \newline
		Proficient in: Python, MATLAB, R, HTML, CSS, \LaTeX, Mathematica \newline
		Familiar with: C++, Julia\\
		
		& \textbf{Softwares and packages} \newline
		Proficient in: Sage, Macaulay2, Qiskit (for quantum computing) \newline
		Familiar with: Tensor Flow, Arduino package from MATLAB, Pupil Labs (eye-tracking devices software) \\
		
		& \textbf{Languages} \newline
		English (fluent), Vietnamese (fluent)
	\end{SectionTable}
	
	% --- Section: Education ---
	
	\begin{SectionTable}{\headingfont Education}
		2020 -- Present & 
		\textbf{University of Illinois at Urbana-Champaign} -- Urbana, Illinois \newline
		PhD in Mathematics \newline 
		Advisors: Professor Philippe Di Francesco and Professor Rinat Kedem %\textit{GPA: 3.89}. 
		\\
		
		2016 -- 2020 & 
		\textbf{St. Olaf College} -- Northfield, Minnesota \newline
		BA in Mathematics with Concentration (minor) in Neuroscience \newline 
		%Mentors: Professors C, D. \textit{GPA: X.YZ}. 
		\\
		
		
		% --- Un-comment the next few lines if you want to include some courses you've taken ---
		
		%& \textbf{Selected coursework}
		%\begin{itemize}[itemsep=0pt, leftmargin=*]
		%\item \textit{Statistics}: Asymptotic statistics, Mathematical statistics, Functional data analysis, High-dimensional statistics, Information theory
		%\item \textit{Mathematics}: Measure theory, Functional analysis, Measure-theoretic probability with martingales
		%\end{itemize}
		
	\end{SectionTable}
	
	% --- Section: Awards, scholarships, etc ---
	
	\begin{SectionTableSingleSpace}{\headingfont Honors and scholarships}
		2022-2023 &
		Bourgin Departmental Fellowship - University of Illinois at Urbana-Champaign\newline
		\\
		
		Summer 2022 &
		R. Ranga and Shantha Rao Scholarships - University of Illinois at Urbana-Champaign  \newline
		\\
		

		2019 &
		\href{https://wp.stolaf.edu/curi/summer-undergraduate-research}{Steen Fellowship} - St. Olaf College \newline 
		\textit{\$4,170  to fund independent summer research project}
		\\
		
		
	\end{SectionTableSingleSpace}
	
	% --- Section: Publications ---
	
	\begin{SectionTable}{\headingfont Publications} 
		2018 & 
		\textbf{Matrix Square Roots of Polynomials} \newline
		Kosmas Diveris, Trung Vu \newline
		\textit{Pi Mu Epsilon Journal}. \\

		2022 & 
		\textbf{T-system with Slanted Initial Data} \newline
		Philippe Di Francesco, Trung Vu \newline
		\textit{[in preparation]}. \\
		
		%2020 & 
		%\textbf{Title of your second most recent research paper} \newline
		%First author, second author, third author, fourth author. \newline
		%\textit{Journal of something or the other}. \\
		
		%2020 & 
		%\textbf{Title of your third most recent research paper} \newline
		%First author, second author, third author, fourth author. \newline
		%\textit{Journal of something or the other}. \\
		
		%2019 & 
		%\textbf{Title of your fourth most recent research paper} \newline
		%First author, second author, third author, fourth author. \newline
		%\textit{Journal of something or the other}.
	\end{SectionTable}
	
	% --- Section: Research experience ---
	
	\begin{SectionTable}{\headingfont Undergraduate Research Experience}
		
		& \textit{\textbf{Joint work at St. Olaf College and University of Illinois at Urbana - Champaign via Steen Fellowship}}\\
		Summer 2019 &
		\textbf{Application of Algebraic Geometry and Geometric Invariant Theory on Functional Neuroimaging} \newline
		Mentor:  Graduate Student Megan Finnegan \\
		& \textit{\textbf{At. St. Olaf College}} \\
		Summer 2018 &
		\textbf{Geographic Variation in Temporal Pattern Recognition in The Acoustic Parasitoid Fly Ormia Ochracea} \newline
		Mentor:  Professor Norman Lee \\
		Fall 2017 -- Sping 2020 &
		\textbf{Free Field Sound Localization Using the Sound Localization Arc} \newline
		Mentor:  Professor Jeremy Loebach \\
		%Project focuses on using SoLoArc (Sound Localization Arc) – a student-made portable sound localization apparatus – to test the ability to localize sound in horizontal space using interaural time difference (ITD) and interaural level difference (ILD), and in vertical space using head-related transfer functions and filtering. We developed the graphical user interface (GUI) for the SoLoArc. The GUI gave students who have had limited coding experience in MATLAB easily access and use the device, allowing it to be used in classes with less supervision. We designed and improved the automated pointing system using a potentiometer and a servo for more precise sound indication.
		Fall 2017 -- Spring 2019  &
		\textbf{Pupillometry and Auditory Cognition in Normal Hearing Listeners, Hearing Impaired Individuals and Cochlear Implant Users} \newline
		Mentor: Professor Jeremy Loebach \\
		%Project investigating auditory and neurocognitive mechanisms that givse rise to accurate speech perception in a variety of listening environments in normal hearing, hearing impaired and cochlear implant users. Responsibility include testing participants in a multipart auditory neurocognitive battery, setting up and running the eye tracker for pupillometry measurements, helping condition and analyzing data. \\

		Summer 2017  &
		\textbf{Matrix Square Roots of Polynomial Project} \newline
		Mentor: Professor Kosmas Diveris. \\
	\end{SectionTable}
	
	% --- Section: Teaching experience ---
	\begin{SectionTable}{\headingfont Teaching}
		& \textit{\textbf{At University of Illinois at Urbana - Champaign}}\\
		Spring 2022 & 
		Teaching Assistant for Calculus 2\\
		
		Fall 2021 & 
		Teaching Assistant for Calculus 1, Ranked as Excellent by Students\\
	
		Spring 2021 & 
		Teaching Assistant for Calculus 2, Ranked as Excellent by Students\\
		
		& \textit{\textbf{At. St. Olaf College}} \\
		Spring 2020 & 
		Teaching assistant for Real Analysis 1 and Combinatorics \\
		Fall 2019 & 
		Supplemental Instructor for Linear Algebra\\
		Spring 2019 & 
		Supplemental Instructor for Linear Algebra\\
		Spring 2018 & 
		Supplemental Instructor for Principles of Statistics\\
		Fall 2017 & 
		Academic Tutor for Calculus 1, Calculus 2 and Linear Algebra\\
		Fall 2017 & 
		Teaching Assistant for General Chemistry  \\
	
		

	\end{SectionTable}
	
	% --- Section: Industry experience ---
	
	%	\begin{SectionTable}{\headingfont Industry experience}
	%		Summer 2020 &
	%		\textbf{Name of company (Title of job or internship)} -- City, State \newline
	%		Description of your responsibilities. Integer pretium semper justo. Proin risus. Nullam id quam. Nam neque. Phasellus at purus et lib ero lacinia dictum.  \\
	
	%		Summer 2019 &
	%		\textbf{Name of company (Title of job or internship)} -- City, State \newline
	%		Description of your responsibilities. Integer pretium semper justo. Proin risus. Nullam id quam. Nam neque. Phasellus at purus et lib ero lacinia dictum.  \\
	
	%		Summer 2018 &
	%		\textbf{Name of company (Title of job or internship)} -- City, State \newline
	%		Description of your responsibilities. Integer pretium semper justo. Proin risus. Nullam id quam. Nam neque. Phasellus at purus et lib ero lacinia dictum.  \\
	%	\end{SectionTable}
	
	% --- Section: Talks and tutorials ---
	

	\begin{SectionTable}{\headingfont Workshops and Conferences}
		April 2022 &
		Analytic Combinatorics in Several Variables Workshop \newline
		\textit{American Institute of Mathematics, San Jose, CA} \\
		
	\end{SectionTable}

	\begin{SectionTable}{\headingfont Talks and Poster Presentations}
		

		& \textit{\textbf{Talks}}\\


		May 2022 &
		Introduction to Analytic Combinatorics in Several Variables with Examples \newline
		\textit{IRT Seminar, University of Illinois at Urbana-Champaign} \\

		March 2022 &
		XXZ Model and Trigonometric $R$-matrix \newline
		\textit{IRT Seminar, University of Illinois at Urbana-Champaign} \\


		February 2022 &
		Introduction to Bethe Ansatz's Equation and the Algebraic Bethe Ansatz\newline
		\textit{IRT Seminar, University of Illinois at Urbana-Champaign} \\

		February 2022 &
		Introduction to Yang-Baxter Equation and Quantum Integrable System\newline
		\textit{IRT Seminar, University of Illinois at Urbana-Champaign} \\
		
		October 2021 &
		T-system with Slanted Initial Data and Pinecone \newline
		\textit{IRT Seminar, University of Illinois at Urbana-Champaign} \\
		

		October 2021 &
		Arctic Curve Phenomenon of T-system via Multivariate Generating Function\newline
		\textit{IRT Seminar, University of Illinois at Urbana-Champaign} \\
		
		May-June 2021 &
		T-system, Dimers and Networks (A series of 5 talks) \newline
		\textit{IRT Seminar, University of Illinois at Urbana-Champaign} \\
		

		February 2021 &
		Introduction to the Pentagram Map, Part 1 - Part 3 \newline
		\textit{IRT Seminar, University of Illinois at Urbana-Champaign} \\
		
		December 2020 &
		Cluster Algebra and Y-patterns \newline
		\textit{IRT Seminar, University of Illinois at Urbana-Champaign} \\
		
		
		October 2019 &
		Matrix Square Roots of Polynomial \newline
		\textit{Northfield Undergraduate Mathematics Symposium, St. Olaf College, Northfield, MN.} \\


		September 2019 &
		Application of Algebraic Geometry and Geometric Invariant Theory on Functional Neuroimaging \newline
		\textit{Steen’s Fellowship Event, St. Olaf College, Northfield, MN.} \\	


		& \textit{\textbf{Poster Presentations}}\\
		
		January 2019 &
		Matrix Square Roots of Polynomial \newline
		\textit{Joint Mathematics Meetings, Undergraduate Poster Session, Baltimore, MD.} \\
		
		May 2018 &
		Pupillometry as A Measure of Auditory Cognitive Processes and Listening Effort. \newline
		\textit{175th Annual Meeting of the Acoustical Society of America, Minneapolis, MN} \\
		
		May 2018 &
		A Comparison of Free-field and Headphone Based Sound Localization Tasks. \newline
		\textit{175th Annual Meeting of the Acoustical Society of America, Minneapolis, MN} \\
		
	\end{SectionTable}
	
	% --- Section: Mentorship and service ---
	
%	\begin{SectionTable}{\headingfont Mentorship and service}
%		\textbf{Title of organization you are in (Name of your role)} \newline
%		Description of your responsibilities. Integer pretium semper justo. Proin risus. Nullam id quam. Nam neque. Phasellus at purus et lib ero lacinia dictum. \\
		
%		Month Year -- Month Year &
%		\textbf{Title of organization you were in (Name of your role)} \newline
%		Description of your responsibilities. Integer pretium semper justo. Proin risus. Nullam id quam. Nam neque. Phasellus at purus et lib ero lacinia dictum. \\
%	\end{SectionTable}
	
	% --- Section: Professional society memberships ---
	
	\begin{SectionTable}{\headingfont Professional Memberships}
		2018 -- Present & 
		Pi Mu Epsilon Mathematical Honor Society \\
		
		2018 -- Present &
		Mathematical Association of America \\
	\end{SectionTable}
	

	
	% --- Section: Other interests/hobbies ---
	
	%	\begin{SectionTable}{\headingfont Other interests}
	%		& 
	%	\end{SectionTable}
	
	% --- End of CV! ---
	
\end{document}