
\documentclass[12pt]{article}
\usepackage{graphicx}
\usepackage{url}
\usepackage{hyperref}
\usepackage{multicol}
\usepackage[dvipsnames]{xcolor}

%% settings to ensure 1 inch margins

%\addtolength{\voffset}{-0.7in}
%\addtolength{\textheight}{1.1in}

%\addtolength{\oddsidemargin}{-0.5in}
%\addtolength{\textwidth}{1in}

%\addtolength{\parskip}{1ex}

%% settings to ensure 1 inch margins




\begin{document}

\begin{center}
\textbf{Math 221 Section CD2/Math 199}
\end{center}

\noindent \textbf{T.A.:} Trung Vu\\				
\textbf{Office Hours:}Monday 12-1pm, Wednesday 3-4pm\\
\textbf{E-mail:} hvu@illinois.edu\\

\noindent \textbf{Merit Discussion Session Grades}\\
Each student may earn up to 10 points for their performance in each Merit session. The points will be assigned on the basis of Attendance, Preparation and Participation.\\
\textbf{Attendance:} I will be lenient with attendance this semester due to COVID. If you feel sick and need to stay at home, no attendance credit will be duducted from  your 10 points, just email me ahead. Classes might be in odd locations so if you need some extra minutes to reach our discussion sections, that is fine.\\
\textbf{Preparation:} You will be expected to come to class having already studied your notes from lecture, having read the text materials assigned for those lectures, and having completed any homework assignments due that day (either assigned by the large lecture instructor or your Merit TA). By studying the material before each class you will be ready to discuss the material in more depth and have specific questions to ask about material that may be giving you difficulty. This will help you immensely during your time in the Merit sessions.\\
\textbf{Participation:}  A large part of how the Merit program will benefit you comes from how you interact with the class.  You are expected to contribute your ideas and insights as well as your questions.  Please use this opportunity to its greatest advantage.\\

\noindent \textbf{The Merit Philosophy}
Merit discussions provide a great opportunity for you to attain a high level of understanding of the course material and to learn to think critically about math in general. The Merit section is not a review session or tutoring session.  We will incorporate aspects of lecture material into the questions that we consider, but this is not meant to be a repeat of lecture nor is it intended to take the place of time you should be studying outside of class.  During this time (where you will be receiving Math 199 credit), you will learn to think critically about complex problems and work with others in small groups to solve them.  You will learn new ways to tackle problems, feel more comfortable about sharing your ideas, make new friends, and realize what you do and don't understand in a comfortable, supportive environment.  You are encouraged to share your opinions in class, but please be aware and respectful that others may have different opinions.  You don't have to agree, but must learn to attack ideas, not people. The biggest benefit of Merit will be gained by students who come to class already familiar with the material and prepared to discuss it, hear new ideas from peers, and apply their understanding to novel situations.  But the format is not a traditional one with an instructor giving information that you are to learn and reproduce. I will be a "non-traditional" instructor for much of the time.  I will often not answer questions directly, but will ask you to consider them further in your groups or to find the information on your own.  We will all need to think in new ways about our roles in this class.  While there is the possibility for frustration inherent in this, I'm confident that we can work together to make this a remarkably valuable experience.\\



\noindent\textbf{Grading Scale:} Your letter grade for Math 199 is determined by the following scale:

\begin{tabular*}{.9\textwidth}{c @{\extracolsep{\fill}}ccccc}
	A&92-100 &A-   &90-91  &  &   \\
	B+ & 88-89 & B & 82-87 & B- & 80-81 \\
	C+ & 78-79 & C & 72-77 & C- & 70-71 \\
	D+ & 68-69 & D & 62-67 & D- & 60-61 \\
	F & 0-60 &  &  &  &  \\
\end{tabular*}


\noindent \textbf{Worksheets and solutions:} Most class periods will focus on worksheets. We will typically spend one hour of class dedicated to working on the regular Math 221 worksheets, and one hour on supplemental Merit problems, though this may be adjusted throughout the term. A copy of the regular Math 221 general worksheets as well as their solutions are provided on the \textcolor{blue}{\href{}{Math 221 Moodle page}} and the merit worksheets will be available on our class webpage \textcolor{blue}{\href{https://htrungvu.com/fall2021_calc1.html}{class website}}. It is the policy in Merit classrooms not to distribute other solutions. I can ask leading questions and help to guide you to an answer during class, or work through problems with you more explicitly during office hours. Sometimes I will go over selected solutions during the discussion section; in this case, I will post my sketch solution to the class webpage later that day.\\


\noindent \textbf{COVID-19 and Classroom Practices:} Following University policy, all students are required to engage in appropriate behavior to protect the health and safety of the community. Students are also required to follow the campus COVID-19 protocols.

Students who feel ill must not come to class. In addition, students who test positive for COVID-19 or have had an exposure that requires testing and/or quarantine must not attend class. A student who is absent for an extended period is encouraged to contact the instructor and obtain a letter from the Student Assistance Center in the Office of the Dean of Students. Absences for these reason will not result in losing participation credit.
   
Students who fail to abide by these rules will first be asked to comply; if they refuse, they will be required to leave the classroom immediately. If a student is asked to leave the classroom, the non-compliant student will be judged to have an unexcused absence and reported to the Office for Student Conflict Resolution for disciplinary action and the class is dismissed. Accumulation of non-compliance complaints against a student may result in dismissal from the University.

\textbf{Face Coverings:} All university students, faculty, staff and visitors must wear a face covering in university spaces indoors, regardless of vaccination status. People who are \textbf{NOT} fully vaccinated are required to wear a face covering in university spaces indoors, as well as outdoors when they cannot practice social distancing. This applies to all students, faculty, staff and visitors. 
	

\noindent \textbf{Questions, Comments, Concerns:} If you are having trouble with any aspect of the course, please seek help as soon as you recognize a problem. You may attend my office hour or set up an appointment at another time. The best way to contact me is via email, and I will try to respond within 24 hours. I will do my best to reply to any message received before 5:00PM that same day. I will not necessarily be able to solve every issue, but I am always willing to listen to your concerns and to try to explain my policies. I want you to succeed, so please let me know if there is something I can do to help with that goal! \\


\emph{I look forward to sharing this discussion with you! If you have any concerns, please contact me.  Have a great semester!}

\end{document}

