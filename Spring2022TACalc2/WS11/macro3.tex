%%%% This  macro file is changed and adapted from professor Jill Dietz's macro from St. Olaf College in MATH320 Linear Algebra 


% This sets the size and style of the page

\documentclass[]{article}
\setlength \oddsidemargin{-.0in} \setlength \textwidth{6.5in}
\setlength \topmargin{-.5in} \setlength \textheight{9in}

% Some packages to install
\usepackage{amsfonts,amssymb,amsmath, mathtools,amsthm}
\usepackage{xcolor, color}
\usepackage{graphicx}
\usepackage{multicol}
\usepackage{url}
\usepackage{faktor}
\usepackage{hyperref}
\usepackage{ulem}
\usepackage{tikz, pgfplots}
\usetikzlibrary{positioning}
\usepackage{pdfpages}
\usepackage{soul}
%\pgfplotsset{compat=1.14}

% various inputs
\input amssym.def
\input amssym.tex



% a good babble textwidth is 5.75in
\newcommand{\babblewidth}{\setlength\textwidth{5.75in}}


% This will stretch out the page
\newcommand{\bigpage}{  \setlength \oddsidemargin{-.25in}
            \setlength \textwidth{6.75in}
            \setlength \topmargin{-1in}
            \setlength \textheight{9.75in}}


%This will shrink the page
\newcommand{\smallpage}{  \setlength \oddsidemargin{.5in}
            \setlength \textwidth{5in}
            \setlength \topmargin{0in}
            \setlength \textheight{9in}}

\newcommand{\separator}{\vglue .1in\hrule\vglue .1in}

\newcommand{\pause}{\vglue .1in\hrulefill {\tiny Pause here}\hrulefill \vglue .1in}

\newcommand{\dashedseparator}{\vspace{.1in}\underline{\hspace{.5in}}\hspace{.5in}\underline{\hspace{.5in}}\hspace{.5in}\underline{\hspace{.5in}}\hspace{.5in}\underline{\hspace{.5in}}\hspace{.5in}\underline{\hspace{.5in}}\hspace{.5in}\underline{\hspace{.5in}}\vspace{.1in}}

\newcommand{\nopages}{\pagestyle{empty}}
\newcommand{\bd}{\begin{document}}
\newcommand{\ed}{\end{document}}
\newcommand{\be}{\begin{enumerate}}
\newcommand{\ee}{\end{enumerate}}
\newcommand{\bi}{\begin{itemize}}
\newcommand{\ei}{\end{itemize}}
\newcommand{\bc}{\begin{center}}
\newcommand{\ec}{\end{center}}
\newcommand{\bq}{\begin{quote}}
\newcommand{\eq}{\end{quote}}
\newcommand{\sub}{\subsection}
\newcommand{\bfi}[1]{\item{\bf{#1}}}
\newcommand{\bfe}[1]{\item{\bf{#1}}}
\newcommand{\bfl}{\begin{flushleft}}
\newcommand{\efl}{\end{flushleft}}
\newcommand{\ba}{\begin{align*}}
\newcommand{\eal}{\end{align*}}
\newcommand{\eqal}[1]{\begin{align*}#1\end{align*}}
\newcommand{\bmc}[1]{\begin{multicols}{#1}}
\newcommand{\emc}{\end{multicols}}
\newcommand{\bp}{\begin{proof}}
\newcommand{\ep}{\end{proof}}


%%general stuff
\newcommand{\caret}{\textasciicircum}

%% Here are some spaces %%
\newcommand{\smsp}{\vspace{.1in}}
\newcommand{\medsp}{\vspace{.25in}}
\newcommand{\half}{\vspace{.5in}}
\newcommand{\inch}{\vspace{1in}}
\newcommand{\quarter}{\vspace{.25in}}

\newcommand{\mytitle}[2]{\begin{center}{\bf {#1}}\\ Trung Vu\\ Collin Nill
{#2}\end{center}}

\newcommand{\centre}[1]{\begin{center}{#1}\end{center}}
\newcommand{\bfcentre}[1]{\begin{center}{\bf {#1}}\end{center}}

\newcommand{\babble}[1]{\marginpar{\flushleft\scriptsize\textsf{#1}}}


% here is highlighted/colored text
%\newcommand{\hl}[1]{\textcolor{red}{#1}} %note that \hl{} highlights text like a highlighter
\newcommand{\hlred}[1]{\textcolor{red}{#1}}
\newcommand{\hlblue}[1]{\textcolor{blue}{#1}}
\newcommand{\hlgreen}[1]{\textcolor{green}{#1}}




%This will put a circle around something.
\newcommand*\circled[1]{\tikz[baseline=(char.base)]{
            \node[shape=circle,draw,inner sep=2pt] (char) {#1};}}

% Theorems and stuff
\theoremstyle{definition}
\newtheorem{theorem}{Theorem}[section]
     \newtheorem{lemma}[theorem]{Lemma}
     \newtheorem{proposition}[theorem]{Proposition}
     \newtheorem{corollary}[theorem]{Corollary}
     \newtheorem{claim}[theorem]{Claim}
     \newtheorem{definition}[theorem]{Definition}
     \newtheorem{example}[theorem]{Example}
     \newtheorem{note}[theorem]{Note}
     \newtheorem{describe}[theorem]{Description}
     \newtheorem{prob}{Problem}
     \newtheorem{question}{Question}[section]
     \newtheorem{remark}{Remark}[section]

     \newenvironment{sol}{{\sc Solution:}}{
~\hfill\rule{2mm}{3mm}\vspace{.1in}}


  %   \newenvironment{proof}{{\sc Proof:}}{
%~\hfill\rule{2mm}{3mm}\vspace{.1in}}



% Command for general
\newcommand{\ds}{\displaystyle}

% Commands for series

\newcommand{\Sumnzero}{\displaystyle{\sum_{n=0}^\infty}}
\newcommand{\Sumnone}{\displaystyle{\sum_{n=1}^\infty}}
\newcommand{\Sumkzero}{\displaystyle{\sum_{k=0}^\infty}}
\newcommand{\Sumkone}{\displaystyle{\sum_{k=1}^\infty}}
\newcommand{\infint}{\int_1^\infty}
\newcommand{\zeroint}{\int_0^1}

% Commands for linear
\newcommand{\LRA}{\Leftrightarrow}
\newcommand{\vectorx}{{\vec x}}
\newcommand{\vectory}{{\vec y}}
\newcommand{\vectorv}{{\vec v}}
\newcommand{\vectorw}{{\vec w}}
\newcommand{\vectorzero}{{\vec 0}}
\newcommand{\vectorb}{{\vec b}}
\newcommand{\vectoru}{{\vec u}}
\newcommand{\vectore}{{\vec e}}
\newcommand{\vectorspan}[1]{{\rm Span}\{#1\}}
\newcommand{\rmnull}[1]{{\rm null}(#1)}
\newcommand{\length}[1]{\parallel #1\parallel}
\newcommand{\vev}{\vectorv}
\newcommand{\veu}{\vectoru}
\newcommand{\vew}{\vectorw}
\newcommand{\vex}{\vectorx}
\newcommand{\vey}{\vectory}
\newcommand{\vea}{\vec{a}}
\newcommand{\veb}{\vec{b}}
\newcommand{\vecc}{\vec{c}}
\newcommand{\vei}{\vec i}
\newcommand{\vej}{\vec j}
\newcommand{\vek}{\vec k}
\newcommand{\vep}{\vec p}
%\newcommand{\vez}{\vectorz}
%\newcommand{\vea}{\vectora}
%\newcommand{\veb}{\vectorb}
\newcommand{\twodvec}[2]{\left[\begin{array}{r}{#1}\\{#2}\end{array}\right]}
\newcommand{\threedvec}[3]{\left[\begin{array}{r}#1\\#2\\#3\end{array}\right]}
\newcommand{\mx}[2]{\left[\begin{array}{#1}#2\end{array}\right]}
\newcommand{\col}[1]{{\rm Col}(#1)}
\newcommand{\xy}{\twodvec{x}{y}}
\newcommand{\xyz}{\threedvec{x}{y}{z}}


%These are two other examples of matrices.
%$G = \bigg\{ \begin{pmatrix} a & b \\ 0 & a \end{pmatrix} \bigg| a,b \in \mathbb{R}, a\neq 0 \bigg\}$ 
%$G = \bigg\{ \begin{bmatrix} a & b \\ 0 & a \end{bmatrix} \bigg| a,b \in \mathbb{R}, a\neq 0 \bigg\}$ 

%Commands for advanced linear
\newcommand{\lvw}{\mathcal{L}(V,W)}
\newcommand{\lvv}{\mathcal{L}(V,V)}
\newcommand{\lv}{\mathcal{L}(V)}
\newcommand{\mt}{\mathcal{M}(T)}
\newcommand{\bs}{\backslash}
%\newcommand{\bf}{\textbf}

% Commands for abstract

\newcommand{\integers}{\mathbb{Z}}
\newcommand{\Z}{\mathbb{Z}}
\newcommand{\K}{\mathbb{K}}
\newcommand{\zee}{\mathbb{Z}}
\newcommand{\reals}{\mathbb{R}}
\newcommand{\R}{\mathbb{R}}
\newcommand{\complex}{\mathbb{C}}
\newcommand{\C}{\mathbb{C}}
\newcommand{\normal}{\triangleleft}
\newcommand{\rationals}{\mathbb{Q}}
\newcommand{\Q}{\mathbb{Q}}
\newcommand{\field}{\mathbb{F}}
\newcommand{\F}{\mathbb{F}}
\newcommand{\N}{\mathbb{N}}
\newcommand{\naturals}{\mathbb{N}}
\newcommand{\aut}[1]{{\rm Aut}(#1)}
\newcommand{\Ker}{{\rm Ker}\,}
\newcommand{\im}{{\rm Im}\,}
\newcommand{\cyclic}[1]{\langle #1 \rangle}
\newcommand{\isom}{\cong}
\newcommand{\autc}[1]{{\rm Aut_c}(#1)}
\newcommand{\autsub}[2]{{\rm Aut}_{#1}(#2)}

\newcommand{\NN}{\mathbb{N}}
\newcommand{\ZZ}{\mathbb{Z}}
\newcommand{\QQ}{\mathbb{Q}}
\newcommand{\RR}{\mathbb{R}}
\newcommand{\CC}{\mathbb{C}}

% Commands for DE

\newcommand{\dydt}{\frac{dy}{dt}}
\newcommand{\dxdt}{\frac{dx}{dt}}
\newcommand{\dPdt}{\frac{dP}{dt}}
\newcommand{\dt}[1]{\frac{d#1}{dt}}
\newcommand{\dFdt}{\frac{dF}{dt}}
\newcommand{\dtwoy}{\frac{d^2y}{dt^2}}
\newcommand{\dYdt}{\frac{d\vey}{dt}}
\newcommand{\p}{\partial}
%%%%%%%% command for graphics %%%%%%%%%%%%%


\newcommand{\graph}[2]{\bc\includegraphics[#1]{#2}\ec}

%%%%%%%%%%%%%%   For letters   %%%%%%%%%%%%%%%%%%%%%%%%%%%%%%%%%%

