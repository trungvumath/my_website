\input macro3
\usepackage{fullpage,amsmath,amssymb,amsthm}

\newcommand{\D}{\displaystyle}

\title{Math 199 CD3 Merit Worksheet 22: Review for Last Midterm: I SWEAR MY QUESTION IS THE EASY ONE. I PROMISE!!!!!!\\
AND I LIKE POLYNOMIAL AND POWER SERIES TOO MUCH. NOTHING IS WRONG WITH THAT}
\date{\today}


\begin{document}
\maketitle
\section{Determine Radius of Convergence}
Calculate $R$, the radius of convergence. If the radius of convergence is infinity, explain why
\begin{enumerate}

	\item $\displaystyle \sum_{n=1}^{\infty} (- 1)^n \frac{ 9n^{6}  x^n }{ 72^n}$
	\vfill
	\item $\displaystyle \sum_{n=1}^{\infty} (- 1)^n \frac{ 10n^{6}  x^n }{ 200^n}$
	\vfill
	\item $\displaystyle \sum_{n=1}^{\infty} (- 1)^n \frac{ 10n^{6}  x^n }{ C^n}$ where $C$ is just any constant. Do you realize something special?
	\vfill
	\newpage
	\item For the following problems, write down the Maclaurin series about $0$ and decide the interval of convergence, radius of convergence and whether the end points are included in the interval of convergence. Binomial Series would be helpful here. I would need you to at least write down the first 3 terms of the binomial coefficients
	\begin{enumerate}
		\item  $\D (1+5x)^{1/2}$
		\vfill
		\item $\D (10+6x)^{1/2}$
		\vfill
		\item $\D (1+3x)^{1/4}$
		\vfill
	\end{enumerate}
\end{enumerate}

\newpage
\section{Calculate the terms of expansion}
\begin{enumerate}
	\item Find the first 3 terms Maclaurin series for $f(x)= \sin^2 x$ about $\pi/4$

	\vfill

	\item Find the first 3 terms Maclaurin series for $f(x)=\frac{x}{\sqrt{1-x^2}}$

	\vfill

	\item Find the first 3 non-zero terms of the Maclaurin series for $xe^{- x}$
	\vfill
\section{Taylor Series}
	\item Let $f(x)=x^3\cos(x^3)$. What is $f^{(21)}(0)$
	\vfill 
	\item Let $f(x)=x^{10}\cos(x^4)$. What is $f^{(18)}(0)$
	The key here is to not actually do 18 and 21 derivative
	\vfill

\end{enumerate}
\section{Other helpful problems}

I can't cover everything, but I highly recommend going through both merit and class worksheet about the Taylor's theorem and how you can manipulate series. Good luck!!
\end{document}